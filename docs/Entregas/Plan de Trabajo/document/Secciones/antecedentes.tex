\section{Antecedentes}
Es fundamental realizar una revisión de investigaciones previas y casos de estudio que hayan abordado tanto la dinámica de los incendios como el uso de técnicas de teledetección y modelado predictivo. La revisión de antecedentes permite situar el presente trabajo en el marco de la literatura especializada, identificar enfoques metodológicos relevantes y reconocer vacíos de conocimiento que motivan la propuesta de investigación. Véase la sección \hyperref[sec:refs]{Referencias}.

La problemática de los incendios forestales no es algo nuevo, su estudio viene siendo desarrollado desde hace varias décadas. En lo ultimos años, el concepto de calentamiento global ha cobrado gran relevancia, y con ello, la necesidad de entender mejor los factores que contribuyen a la ocurrencia y propagación de incendios forestales. El avance tecnológico, especialmente en el ámbito de la teledetección y el análisis de datos, ha abierto nuevas posibilidades para monitorear y predecir estos eventos.

Diversos estudios han demostrado la eficacia de utilizar datos satelitales para detectar áreas afectadas por incendios y evaluar su extensión. Además, la integración de datos climáticos, topográficos y de vegetación ha permitido desarrollar modelos predictivos más precisos. En particular, el uso de algoritmos de aprendizaje automático, como las redes neuronales, ha mostrado un gran potencial para mejorar la capacidad de predicción de incendios forestales.

