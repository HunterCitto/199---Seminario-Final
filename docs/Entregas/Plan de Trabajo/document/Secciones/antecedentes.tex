\section{Antecedentes}
La problemática de los incendios forestales no es algo nuevo, su estudio viene siendo desarrollado desde hace varias décadas. En lo ultimos años, con el concepto de calentamiento global que ha cobrado gran relevancia, y con ello, la necesidad de entender mejor los factores que contribuyen a la ocurrencia y propagación de incendios forestales. El avance tecnológico, especialmente en el ámbito de la teledetección y el análisis de datos, ha abierto nuevas posibilidades para el estudio de estos eventos.

Diversos estudios han demostrado la eficacia de utilizar datos satelitales para detectar áreas afectadas por incendios y evaluar su extensión. Además, la integración de datos climáticos, topográficos y de vegetación ha permitido desarrollar modelos predictivos más precisos. \\

El trabajo llevado a cabo en la Universidad del Norte (Barranquilla, Colombia) \hyperref[sec:refs]{[1]} es un ejemplo destacado de cómo la combinación de datos históricos, meteorologicos, topográficos y socioeconómicos fueron utilizados como entrada a modelos predictivos con la finalidad de entender su eficacia en la predicción y proporcionar información valiosa para investigaciones futuras. \\

Por otro lado, en el trabajo realizado en la Universidad Forestal de Beijin (Pekín, China) \hyperref[sec:refs]{[2]} han utilizando un tipo especifico de red neuronal recurrente (RNN) denominada LSTM (Long Short-Term Memory) para predecir la escala de incendios forestales en función de registros históricos de focos de incendios (Canada National Fire Database) similar a la información aportada por la NASA a través de FIRMS y el CONAE por medio de su servicio consultas interactivas de focos de calor y áreas quemadas, sumando un total de 11 variables meteorologicas. Como resultado han demostrado la factibilidad de predecir la escala de incendios forestales. \\

Los articulos \hyperref[sec:refs]{[3]} y \hyperref[sec:refs]{[4]} presentan una revisión de modelos de Machine Learning con aplicación en el tema. El primero de ellos hace un análisis comparativo entre diferentes modelos de ML y brinda ejemplos de aplicación en distintos objetivos relacionados a la problemática mientras que el segundo brinda otro ejemplo de implementación y rendimiento en modelos de ML aplicados.\\

Por el lado de \hyperref[sec:refs]{[5]} y \hyperref[sec:refs]{[6]} se presentan trabajos similares dentro de un contexto nacional (Córdoba y Pinamar) que si bien son acotados en contenido, comparten el espritu de la investigación y el desarrollo de modelos predictivos para la problemática de los incendios forestales.