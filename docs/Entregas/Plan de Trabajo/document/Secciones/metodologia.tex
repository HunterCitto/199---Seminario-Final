\section{Actividades y metodología}
\subsection{Enfoque Metodológico}
El presente trabajo adoptará un enfoque mixto con tendencia al cuantitativo, bajo la metodología CIRSP-DM (Cross-industry Standard Process for Data Mining). Se seguirán los principios del análisis exploratorio de datos (entendimiento), tratamiento de datos satelitales (preparación) y técnicas de Machine Learning (modelado). Además, se integrarán marcos conceptuales relacionados con el manejo del fuego y la gestión de riesgos ambientales.
\subsection{Resumen de Actividades}
\begin{itemize}
    \item \textbf{Entendimiento:} Análisis de literatura académica y técnicas desarrolladas en materia de incendios forestales, modelos predictivos y uso de datos satelitales (FIRMS, CONAE, Copernicus). \\ 
    \item \textbf{Preparación de datos:} Descarga y preprocesamiento de datos satelitales de fuentes como: FIRMS (NASA), Copernicus (UE) y la CONAE (ARG), junto con reportes del Servicio Nacional de Manejo del Fuego. Limpieza, normalización y estructuración. \\
    \item \textbf{Análisis, modelado y evaluación:} 
        \begin{enumerate}
            \item Análisis exploratorio de datos (EDA) y visualización geoespacial.  
            \item Identificación de focos de calor y zonas críticas mediante la aplicación de técnicas de clustering y detección de anomalías.  
            \item Identificación predictivo con algoritmos de Machine Learning aún por definir (en evaluación: Random Forest, SVM, Gaussian Mixture Models, XGBoost, Redes Neuronales Convolucionales).
            \item Evaluación de modelos mediante métricas como precisión, recall, F1-score y AUC-ROC. \\
        \end{enumerate}
    \item \textbf{Elaboración de informe:} Recopilación de resultados y redacción del documento final.  
\end{itemize}