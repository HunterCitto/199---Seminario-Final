\section{Actividades y metodología}
NOTA: Enumerar algunas tareas a desarrollar y las metodologías experimentales y técnicas a emplear en el plan de trabajo propuesto para la obtención de resultados. Pueden consignar herramientas técnicas de análisis de datos vistos en la carrera también utilizar marcos conceptuales tomados de las materias cualitativas. (BORRAR)

\subsection{Enfoque Metodológico}
Pendiente de definición.

\subsection{Resumen de Actividades}
\begin{itemize}
    \item \textbf{Revisión bibliográfica:} Análisis de literatura académica y técnicas desarrolladas en materia de incendios forestales, modelos predictivos y uso de imágenes satelitales (FIRMS, CONAE, Copernicus, MODIS, Sentinel). \\ 
    \item \textbf{Recolección de datos:} Descarga y preprocesamiento de datos satelitales históricos y actuales de fuentes como: FIRMS (NASA), Copernicus (UE) y la Comisión Nacional de Actividades Espaciales (CONAE), junto con reportes del Servicio Nacional de Manejo del Fuego. \\
    \item \textbf{Procesamiento:} Limpieza, normalización y estructuración para generar un repositorio de datos históricos que permita análisis comparativos. \\
    \item \textbf{Análisis:} 
        \begin{enumerate}
            \item Análisis exploratorio de datos (EDA) y visualización geoespacial.  
            \item Identificación de focos de calor y zonas críticas mediante la aplicación de técnicas de clustering y detección de anomalías.  
            \item Modelado predictivo con algoritmos de Machine Learning aún por definir (en evaluación: Random Forest, SVM, Gaussian Mixture Models, XGBoost, Redes Neuronales Convolucionales para imágenes).  \\
        \end{enumerate}
    \item \textbf{Elaboración de informe:} Sistematización de resultados, generación de visualizaciones y redacción del documento final.  
\end{itemize}