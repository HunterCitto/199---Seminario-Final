\section{Actividades y metodología}
\subsection{Enfoque Metodológico}
El presente trabajo adoptará un enfoque mixto con tendencia al cuantitativo, bajo el paradigma de la Ciencia de Datos aplicada a problemas ambientales. Se seguirán los principios del análisis exploratorio de datos (EDA), técnicas de Machine Learning y metodologías de teledetección para el tratamiento de imágenes satelitales. Además, se integrarán marcos conceptuales relacionados con el manejo del fuego y la gestión de riesgos ambientales.
\subsection{Resumen de Actividades}
\begin{itemize}
    \item \textbf{Revisión bibliográfica:} Análisis de literatura académica y técnicas desarrolladas en materia de incendios forestales, modelos predictivos y uso de imágenes satelitales (FIRMS, CONAE, Copernicus, MODIS, Sentinel). \\ 
    \item \textbf{Recolección de datos:} Descarga y preprocesamiento de datos satelitales históricos y actuales de fuentes como: FIRMS (NASA), Copernicus (UE) y la Comisión Nacional de Actividades Espaciales (CONAE), junto con reportes del Servicio Nacional de Manejo del Fuego. \\
    \item \textbf{Procesamiento:} Limpieza, normalización y estructuración de datos con el fin de generar un repositorio de datos temporales y geolocalizados que permita análisis comparativos. \\
    \item \textbf{Análisis:} 
        \begin{enumerate}
            \item Análisis exploratorio de datos (EDA) y visualización geoespacial.  
            \item Identificación de focos de calor y zonas críticas mediante la aplicación de técnicas de clustering y detección de anomalías.  
            \item Modelado predictivo con algoritmos de Machine Learning aún por definir (en evaluación: Random Forest, SVM, Gaussian Mixture Models, XGBoost, Redes Neuronales Convolucionales para imágenes).  \\
        \end{enumerate}
    \item \textbf{Elaboración de informe:} Recopilación de resultados, generación de visualizaciones y redacción del documento final.  
\end{itemize}