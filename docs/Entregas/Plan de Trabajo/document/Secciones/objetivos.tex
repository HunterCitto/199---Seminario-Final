\section{Objetivos}
Por medio de esta investigación se busca abordar el fenómeno de incendios forestales en la Patagonia Andina a partir de un enfoque basado en datos e imágenes satelitales. A continuación, se establecen los objetivos que darán lugar al desarrollo del trabajo, orientados tanto a la comprensión de los factores asociados a la aparición de focos de calor como al diseño de herramientas que permitan su detección y predicción.

\subsection{Objetivos Generales}
\begin{itemize}
    \item Estudiar y comprender los incendios forestales de la Patagonia Andina que tienen lugar durante epocas de verano.
    \item Analizar los factores determinantes, evaluando su relación con la ocurrencia de incendios y explorar su potencial en la construcción de modelos predictivos.
\end{itemize}

\subsection{Objetivos Específicos}
\begin{itemize}
    \item Recopilar y sistematizar la información necesaria y disponible para el desarrollo del proyecto, incluyendo imágenes satelitales, datos climatológicos y ambientales.
    \item Implementar modelos de Machine Learning que permitan identificar focos activos y predecir la probabilidad de nuevos focos en una región específica.
    \item Evaluar la efectividad de los modelos desarrollados mediante la comparación de sus predicciones con datos históricos de incendios.
\end{itemize}