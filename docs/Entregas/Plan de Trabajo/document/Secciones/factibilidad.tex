\section{Factibilidad}
La factibilidad del este trabajo se sustenta en la disponibilidad de datos abiertos, herramientas tecnológicas de libre acceso y bibliografía especializada en el área de estudio. También es considerada la capacidad computacional necesaria (ya sea entorno local o uso de recursos en la nube) para llevar a cabo los análisis planteados. \\

El cronograma de trabajo será acotado pero realista, su diseño estará pensado para ser ejecutado en un plazo máximo de dos meses.

\subsection{Recursos Disponibles}
\begin{itemize}
    \item \textbf{Datos:} Acceso a imágenes satelitales provenientes de programas internacionales como Copernicus (UE) y FIRMS (NASA), disponibles mediante APIs y portales oficiales de libre acceso (CONAE y otros nacionales).
    \item \textbf{Software:} Herramientas de análisis de datos y programación en lenguaje Python, complementando con librerías de uso extendido en ciencia de datos (pandas, scikit-learn, geopandas, entre otras).
    \item \textbf{Bibliografía:} Acceso a literatura académica, artículos y reportes técnicos disponibles en internet.
    \item \textbf{Infraestructura:} Equipo computacional personal con capacidad suficiente para el procesamiento de imágenes y ejecución de modelos de aprendizaje automático. En su defecto o a fines de optimizar tiempos, se montará una infraestructura virtual en un Droplet de DigitalOcean.
\end{itemize}

\subsection{Cronograma de Actividades}
\begin{table}[h]
\centering
\begin{tabular}{lccl}
\toprule
\textbf{Actividad} & \textbf{Inicio} & \textbf{Fin} & \textbf{Entregable} \\
\midrule
Revisión bibliográfica & 23/08 & 06/09 & Plan de Trabajo \\
Recolección y preparación de datos & 06/09 & 13/09 & Dataset completo y procesado \\
Análisis preliminar & 13/09 & 20/09 & Resultados iniciales \\
Modelado y evaluación & 20/09 & 11/10 & Hallazgos principales \\
Redacción final & 11/10 & 24/10 & Informe completo \\
\bottomrule
\end{tabular}
\caption{Cronograma de trabajo (aproximación 10 semanas total con margen)}
\end{table}

\subsection{Factibilidad Temporal}
El plan de trabajo ha sido diseñado para su desarrollo en un plazo máximo de dos meses, con actividades secuenciales que permiten avanzar de manera progresiva desde la revisión bibliográfica hasta la redacción del informe final. Si bien los tiempos son ajustados, la disponibilidad de datos abiertos, software libre y bibliografía especializada garantizan la viabilidad del proyecto dentro de los plazos académicos establecidos.